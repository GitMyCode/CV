% Exemple de CV utilisant la classe moderncv
% Style casual en orange
% Article complet : http://blog.madrzejewski.com/creer-cv-elegant-latex-moderncv/

\documentclass[a4paper,10pt]{extarticle}

%\usepackage{fourier-orns} \Huge\hrulefill\hspace{0.2cm} \floweroneleft\floweroneright \hspace{0.2cm} \hrulefill
%\usepackage{fancyhdr} \pagestyle{fancy}  \lhead{\hrule \vbox to 10pt{\vrule height 05pt width 1pt} LEFT}

%\usepackage[factor=1100,stretch=10,shrink=10]{microtype}
\usepackage{textcase}
\usepackage[none]{hyphenat}
\usepackage[thinlines]{easytable}
\usepackage{comment}
\usepackage[utf8]{inputenc}
\usepackage[french]{babel}
\inputencoding{latin1} % important pour les carathere francais

\usepackage[left=1cm,right=1.5cm,top=1.5cm,bottom=0.5cm]{geometry}
%\usepackage[top=2.1cm, bottom=1.1cm, left=2cm, right=2cm]{geometry}

% Largeur de la colonne pour les dates
%A Few Useful Packages
%\usepackage{marvosym}\textsc{\textsc{}}
\usepackage{fontspec} 					%for loading fonts
\usepackage{xunicode,xltxtra,url,parskip} 	%other packages for formatting
\RequirePackage{color,graphicx}
\usepackage[usenames,dvipsnames]{xcolor}
%\usepackage[big]{layaureo} 				%better formatting of the A4 page
% an alternative to Layaureo can be ** \usepackage{fullpage} **
%\usepackage{supertabular} 				%for Grades
\usepackage{titlesec}					%custom \section
\usepackage{ifthen} % pour les ifthenelse
\usepackage{pbox} % pour faire un \\ dans une cellule
\usepackage{setspace} % pour les begin{spacing}{0/7}

\usepackage{mathptmx}
%\usepackage{lmodern}
\usepackage[T1]{fontenc}

\usepackage{fontawesome}
%-----------Font



\AtBeginDocument{\def\labelitemi{$\bullet$}} %pour redefinir les bulletlist
%Setup hyperref package, and colours for links
\usepackage{hyperref}
\definecolor{linkcolour}{rgb}{0,0.2,0.6}
\hypersetup{colorlinks,breaklinks,urlcolor=linkcolour, linkcolor=linkcolour}
%----------------------------------------------

\usepackage{enumitem}

\titleformat{\section}{\Large\mdseries\upshape}{}{0em}{}[\titlerule]

\usepackage[absolute]{textpos}
\setlength{\TPHorizModule}{30mm}
\setlength{\TPVertModule}{\TPHorizModule}
\textblockorigin{2mm}{0.65\paperheight}
\setlength{\parindent}{0pt}


%------------Definition de column custom-------------
\usepackage{array}
\newcolumntype{C}[1]{>{\centering}p{#1}}
\newcolumntype{L}[1]{>{\raggedleft}p{#1}}
\newcolumntype{R}[1]{>{\raggedright}p{#1}}
%------------             FIN           -------------


\newcommand{\cventry}[7][.25em]{}
\renewcommand{\cventry}[7][.25em]{%
  \cvitem[#1]{#2}{%
    {\bfseries #3}%
    \ifthenelse{\equal{#4}{}}{}{, {\slshape #4}}%
    \ifthenelse{\equal{#5}{}}{}{, #5}%
    \ifthenelse{\equal{#6}{}}{}{, #6}%
    .\strut%
    \if x& #7 &%
      \else{\newline{}\begin{minipage}[t]{\linewidth}\small #7 \end{minipage}}\fi}}




% colors
%-------
\definecolor{color0}{rgb}{0,0,0}% main default color, normally left to black
\definecolor{color1}{rgb}{0,0,0}% primary scheme color
\definecolor{color2}{rgb}{0,0,0}% secondary scheme color
\definecolor{color3}{rgb}{0,0,0}% tertiary scheme color

%lengths width
\newlength{\indicewidth}%
\setlength{\indicewidth}{0.13\textwidth}%45pt0.175\textwidth
\newlength{\separatorcolumnwidth}%
\setlength{\separatorcolumnwidth}{0.025\textwidth}%
\newlength{\maincolumnwidth}%
\setlength{\maincolumnwidth}{0.84\textwidth}%
\newlength{\spacecvline}%
\setlength{\spacecvline}{0.85em}
%-----------FIN def lenght


% Styles
\newcommand{\hintfont}{}
\newcommand{\indicestyle}[1]{\slshape\textcolor{color0}{#1}}


\renewcommand{\labelitemii}{\small$\blacksquare$} 
%%%%%%%%%%%%%% NEW COMANDE %%%%%%%%%%%%%%


%CVITEM----------------------------------
\newcommand{\cvitem}[3][\spacecvline]{
\noindent\begin{tabular}
{@{}p{\indicewidth}@{\hspace{\separatorcolumnwidth}} p{\maincolumnwidth}@{}}
	\raggedleft{\emph{#2}} &{ #3}%
\end{tabular}%
\vspace{-1em}\par\addvspace{#1}  }


%CVDOUBLEITEM-----------------------------
\newcommand{\cvdoubleitem}[5][\spacecvline]{%

 \cvitem[#1]{\bf #2}{%
   \begin{minipage}[t]{0.4\linewidth}#3\end{minipage}%
   %
   \begin{minipage}[t]{\indicewidth}\bf\raggedleft{#4}\end{minipage}%
   \hspace*{\separatorcolumnwidth}
   \begin{minipage}[t]{0.4\linewidth}#5\end{minipage}}
  
}


%CVENTRE----------------------------------
\newcommand{\cventre}[6][\spacecvline]{
\cvitem[#1]{#2}{
	{\bfseries #3}%
	\ifthenelse{\equal{#4}{}}{}{, {\slshape #4}}\strut\vspace{0.2em}%
	\ifx&#5&
	    \else{\vspace{-0.1em}\newline{}\begin{minipage}[t]{\linewidth}\begin{spacing}{0.8} #5 \end{spacing} \end{minipage}}\fi
	\ifx&#6&
	    \else{\vspace{-0.5em}\newline{} \hspace*{1.7em}\begin{minipage}[t]{\linewidth}\small\upshape #6 \end{minipage}}\fi
}}



%CVDETAILITEM-----------------------------
\newcommand{\cvdetailitem}[3][\spacecvline]{
\noindent\begin{tabular}
{@{}p{\indicewidth}@{\hspace{\separatorcolumnwidth}} p{\maincolumnwidth}@{}}
	\raggedright{\bf{#2}} &{ #3}%
\end{tabular}%
\vspace{-0.9em}\par\addvspace{#1}  }

%CVDETAILENTRE----------------------------
\newcommand{\cvdetaillist}[3][.25em]{}
\renewcommand{\cvdetaillist}[3][.25em]{%
  \cvdetailitem[#1]{#2}{#3}}



%CVBOXITEM--------------------------------
\newcommand{\cvboxitem}[5]{}
\renewcommand{\cvboxitem}[5]{%
	\begin{minipage}[t]{\linewidth}
	
	$\bullet$ \textbf{#1}
	\ifthenelse{\equal{#2}{}}{}{, {\slshape #2}}\par%	
	\raggedright{\scriptsize\href{#3}{#3}}\\
	{\small\slshape\color{Bittersweet} {#4}}\vspace{0.2em}\\
	{#5}

	\end{minipage}
	
}
  

%CVTRIPLEBOX------------------------------
\def\cvtripleboxutil#1#2#3#4{
\cvboxitem{#1}{#2}{#3}{#4}
}
%CVTRIPLEBOX------------------------------
\newcommand{\cvtriplebox}[3]{}
\renewcommand{\cvtriplebox}[3]{
\begin{tabular*}{\linewidth}{  
                   p{\dimexpr 0.33\linewidth-1\tabcolsep } 
                   p{\dimexpr 0.33\linewidth-1\tabcolsep }  
                   p{\dimexpr 0.33\linewidth-1\tabcolsep } 
                   } 
{#1}&{#2}&{#3}\tabularnewline
\end{tabular*}
}




%SUBSECTION----------------------------------
\def\middleline{
\raisebox{0.35em}{\line(1,0){50}}
}

\renewcommand{\subsection}[1]{
\par\addvspace{3ex}\hspace{-2em}
\begin{tabular}{@{}p{\indicewidth}@{\hspace{\separatorcolumnwidth}}p{\maincolumnwidth}@{}}%
    \raggedleft\indicestyle{}\middleline & { \strut\bfseries {#1} }%
    \par
\end{tabular}%
\vspace{-0.5em}
}



\long\def\/*#1*/{}
\usepackage{pdfpages}


%%%%%%%%%%%%%%  FIN New COMANDE  %%%%%%%%%%%%%% 

 \renewcommand\familydefault{\sfdefault}



%-------------------- DEBUT DU DOCUMENT --------------------------------------------------%
%-------------------- DEBUT DU DOCUMENT --------------------------------------------------%
%-------------------- DEBUT DU DOCUMENT --------------------------------------------------%
\begin{document}



\begin{table}[ht]
\begin{minipage}[b]{0.74\textwidth}
\begin{tabular}{l}

\fontsize{43}{36}\textsf{Martin Bouchard}%
 \vspace{0.6em}\\\LARGE{\textit{ Programmeur }}
\end{tabular}
\end{minipage}
\hspace{0.5cm}
\noindent
\begin{minipage}[b]{0.25\textwidth}\raggedright
\begin{tabular}{|r}
\\
  \textit{4449 rue Adam}  \\
   \textit{Montréal H1V 1T7}\\ 
  \textit{marbouchl@gmail.com} \\
   \textit{438 888 9741}  \\
    \faGithub\hspace{0.2em}\textit github: GitMyCode 
    
\end{tabular}
\end{minipage}
\end{table}


\section{Sommaire}
{Diplômé d'un baccalauréat en programmation et génie logiciel. Je perfectionne et étends également mes connaissances de manière autodidacte dans mes temps libres. J'ai un intérêt pour les problèmes algorithmiques ( particulierement ceux sur \\\href{https://code.google.com/codejam}{code.google.com/codejam}) et les applications web. J'ai de l'expérience dans l'ensemble du processus de développement logiciel, de la conception jusqu'à la mise en production et la maintenance.}
\vspace{-1em}
\section{Formation}

\cventre{2011-2014}{Baccalauréat informatique}{Université du Québec à Montréal}{}{}
\cventre{2010}{Certificat en Criminologie}{ Université de Montréal}{}{}

\cventre{2008}{Attestation Études Cinématographiques}{École de Cinéma et de télévision de Québec}{}{}



\section{Expérience Professionnelle}

\cventre[2em]{2014}{Stagiaire (Développeur-web) }{Telica Software}
					 {Développeur back-end et front-end sur une application web en Ruby on Rails }
					 {$\bullet$ TDD $\bullet$ Développement Agile $\bullet$ Architecture MVC $\bullet$ Git $\bullet$ Ruby on Rails $\bullet$ Rspec/test \\ $\bullet$ Déploiement code production  }

\cventre[1.2em]{2014}{Développer Web}{Contrat}{ Un re-design d'une page web pour un avocat. Le client voulait un design plus modern qui serait \emph{ responsive design } pour les visiteurs mobile. \href{http://www.avocatcriminel.ca/}{www.avocatcriminel.ca}}


\cventre[2em]{2013}{Stagiaire (Développeur)}
{Ville de Montréal}
{Conception d'applications d'affaires sur la plateforme de développement web Apex de Oracle.}
{$\bullet$ Design d'interface utilisateur  $\bullet$ Programmation front-end et back-end(Apex) de page web \\$\bullet$ Conception de schema bd  $\bullet$ Intégration avec LDAP $\bullet$ Architecture d'application sous forme de service}


\cventre[0.5em]{2012}{Programmeur}
{Fasken Martineau DuMoulin, S.E.N.C.R.L}
{Programmation de templates avec l'API de XpressDox. Cette API s'intègre avec Microsoft Word et automatise la génération de document. J'étais le seul qui s'avais utiliser Xpressdox dans l'entreprise.}
{$\bullet$ Développer un framework $\bullet$ Proposer des solutions techniques aux besoins des avocats \newline{} $\bullet$ Trouver des solutions à des problèmes sans ressource}
 

\section{Compétence Informatique}
\vspace{1em}
%\begin{comment}
\cvdoubleitem{Programmation}{Java, Scala, Python, C/C++, Assembleur, BASH, Ruby}
			 {System}{Windows, Linux}\vspace{1em}
\cvdoubleitem{Web framework}{Apex, Django, Wordpress, Ruby on Rails, Play}
			 {Web}{HTML/CSS , PHP, javascript, jQuery Mobile, AngularJS, Bootstrap  }\vspace{1em}
\cvdoubleitem{BD}{MySQL, Oracle, SQL, PL/SQL, JPA , JDBC}
			 {Génie}{UML, Design Patterns, Agile, Git}\vspace{1em}
\cvdoubleitem{Outils Développement}{Vim, SublimeText, RubyMine, Intellij, Android Studio}
			 {Language Markup}{\LaTeX{}, Markdown}\vspace{1em}

\/*
\begin{tabular}{ll}
\textbf{Langages} & 
html/css/php/javascript, jQuery Mobile, Apex, PL/SQL, C/C++, Java, Bash, Python,    Latex, Assembleur  \\
\textbf{Génie} &
UML, Design Patterns\\
\textbf{BD} & MySQL, Oracle \\
\textbf{CMS} & WordPress, Django \\
\textbf{Systèmes} & Windows, Linux \\
\textbf{IDE} & Eclipse, vim, SublimeText, Photoshop
\end{tabular}
*/
%\end{comment}
\begin{comment}
\begin{tabular*}{\linewidth}{
				   p{\dimexpr 0.15\linewidth-1\tabcolsep }  
                   p{\dimexpr 0.25\linewidth-1\tabcolsep } 
                   p{\dimexpr 0.25\linewidth-1\tabcolsep }  
                   p{\dimexpr 0.25\linewidth-1\tabcolsep } 
                   } 
   		         & Programmation & Web/Framework & BD \vspace{1.5em}\tabularnewline
À l'aise		 &               &   Django,WordPress            & \tabularnewline[10ex]
Très à l'aise 	 &               &   Ruby on rails           & \tabularnewline[10ex]
Mon best         & Java \newline C\#  & & \tabularnewline[10ex]
\end{tabular*}
\end{comment}
\clearpage














\section{Travail Pratique \small (École)}
\begin{tabular*}{\textwidth}{p{\dimexpr 0.15\linewidth-1\tabcolsep }
						 	 p{\dimexpr 0.65\linewidth-1\tabcolsep }} 
\centering\underline{Langage} &\centering\underline{Description}
\end{tabular*}\vspace{-1em}
\\\\

\cvdetaillist{Java}{
\begin{itemize}\setlength{\itemsep}{1pt}
  \setlength{\parskip}{5pt}
  \setlength{\parsep}{0pt}
\item Jeu de Puissance4 avec un A.I. minmax et une interface. Le jeu a été conçu dans le cadre d'un cours sur les designs pattern. Il y a notamment: Strategy, Command, Memento, Factory, Facade \small \href{https://github.com/GitMyCode/puissance4}{github.com/GitMyCode/puissance4}
\item Intelligence artificielle du jeu Connect5(\textit{Go-Moku}). Tous les A.I. de la classe devaient participer à un tournois entre eux. La note était attribuée en fonction du placement après la compétition. Mon A.I. s'est distingué des autres et a obtenu la première place   \small \href{http://gitmycode.github.io/Connect5-Ai/}{gitmycode.github.io/Connect5-Ai} 

\item Solveur du jeu \emph{Sokoban} avec l'algorithme A*  \small \href{https://github.com/GitMyCode/Sokoban-solver}{github.com/GitMyCode/Sokoban-solver}


\end{itemize}
}

\cvdetaillist{C/C++}{
\begin{itemize}\setlength{\itemsep}{1pt}
  \setlength{\parskip}{5pt}
  \setlength{\parsep}{0pt}
\item Implémentation des fonctionnalitées de la commande \emph{find} d'Unix
\item Simulation de l'algorithme de remplacement des pages (Optimial, Horloge, LRU). Conception d'un tableau dynamique 2 dimensions.

\item Jeu de \emph{snake} avec une implémentation d'un tableau dynamique
\item Système interrogeable prenant en entrée des données de location de personnes. Le temps d'éxécution du programme devait être optimisé \emph{(meilleur temps de la classe)}. Implémentation d'un Arbre-map
\item Système de recherche dans un Graph. Implémentation de l'algorythme de Tarjan et Dijkstra
\end{itemize}
}
\cvdetaillist{\small Nodejs/Mongodb HTML/CSS}{$\bullet$ Application web couvrant la plupart des verbes http sur une base de données mongodb   \newline\small{\href{http://gitmycode.github.io/inf4375-tp3/}{http://gitmycode.github.io/inf4375-tp3/}}}
\vspace{1em}\par










\vspace{-0.5em}
\section{Réalisation Technique \small(projets personnels)}

\cvtriplebox
{\cvboxitem{MineSweeper AI}{}{https://github.com/GitMyCode/Minesweeper-AI}{Java}
{Un projet initialement personnel qui a été repris pour un projet de fin de session d'un cours d'Intelligence Artificielle}}
{\cvboxitem{Encefal}{projet en collaboration}{http://foireauxlivres.uqam.ca/}{Django}
{La foire aux livres permet aux etudiants de vendre leurs livres à d'autres étudiants.}}
{
\cvboxitem{Site web}{abandonné}{www.courspiano.ca}{Wordpress,PHP}
{Un site construit sur la plateforme Wordpress. Design fait sur photoshop puis realisé en theme-wordpress}
}



\cvtriplebox
{\cvboxitem{Programming Mobile Applications for Android Handheld Systems}{}{https://class.coursera.org/android-001}{Android, Java}
{J'ai suivi un cours de développement sur la plateforme Android. C'est un cours en ligne donné par l'Université du Maryland via le site coursera.org}
}
{\cvboxitem{Site web}{abandonné}{http://coursmusique.codefaction.webfactional.com}{Django}
{Un site designé sur photoshop et transposé en html/css.}}
{
\cvboxitem{Site Web}{contrat}{http://gitmycode.github.io/avocatmontreal/}{ HTML/CSS/Javascript}
{Un contrat de mise à jours d'un siteweb vers un design plus moderne et mobile friendly }
}
\cvtriplebox
{\cvboxitem{Android Game}{en cours}{https://github.com/GitMyCode/laser-game}{Unity,C\#}{Une petite idée de jeu android que j'ai eue. C'est principalement pour le plaisir de le faire plus que pour le jeu. C'est un jeu en 2d real-time multiplayer}}
{\cvboxitem{Google Codejam}{}{https://github.com/GitMyCode/code-jam}{Java, Scala, Dynamic programming}
{\href{https://code.google.com/codejam}{Google codejam} est un site de compétition de programmation. Ce \emph{repo} contient mes solutions. Le code n'est malheureusement pas très entretenu }}

{}



\vspace{-2em}
\section{Autre}\vspace{-2.5em}
\begin{comment}
\begin{tabular*}{\linewidth}{  
                   p{\dimexpr 0.33\linewidth-1\tabcolsep } 
                   p{\dimexpr 0.66\linewidth-1\tabcolsep }  
                   }
\begin{minipage}[t]{\linewidth}                   
\subsection{Langue}
\begin{itemize}
\item[$\bullet$] Français langue maternelle 
\item Anglais fonctionnel
\end{itemize}
\end{minipage}
&
\begin{minipage}[t]{\linewidth}
\subsection{Intérêt}
\begin{itemize}
\item[$\bullet$] Programmation : Toujours à la recherche de projets intéressants !
\item Guitare
\end{itemize}
\end{minipage}
\tabularnewline

\end{tabular*}
\end{comment}


%\begin{comment}
\cvtriplebox{
\subsection{Langue}
\begin{itemize}
\item[$\bullet$] Français langue maternelle 
\item Anglais fonctionnel
\end{itemize}
}
{
\subsection{Intérêt}
\begin{itemize}
\item[$\bullet$] Programmation : Toujours à la recherche de projets intéressants !
\item Guitare
\end{itemize}
}
{
\subsection{Implications sociales}
\begin{itemize}
\item[$\bullet$] Associations étudiantes
\item CSGAMES
\end{itemize}
}
%\end{comment}
%\includepdfmerge{ReleverNotesNonOfficiel.pdf,-}
\end{document}
